% Se incluye el preambulo del trabajo
\input{./preamble.tex}

\begin{document}

	% Crear y configurar el titulo/caratula del informe
	\title{
		\normalfont \normalsize \textsc{Instituto Tecnol\'ogico de Buenos Aires} \\ [25pt]
		\huge FilterTool : Manual de usuario \\
		\author{
			\\Grupo 1:\\\\Farall, Facundo David\\Gaytan, Joaqu\'in Oscar\\Kammann, Lucas Agust\'in\\Maselli, Carlos Javier		 
		}
		\text{Teor\'ia de Circuitos - 2019}
	}
	\pagenumbering{arabic}
	\maketitle
	\newpage
\newpage

	% Se agrega el indice con el contenido del trabajo
	\tableofcontents
	\newpage


\section{Introducci\'on y generalidades}
FilterTool es una herramienta que permite realizar el dise\~no completo de filtros activos,desde la platilla hasta el circuito de implementaci\'on, en base a aproximaciones conocidas y por medio de una interfz gr\'afica. 
Se presenta a continuaci\'on una imagen de la interfaz gr\'afica b\'asica.
\begin{figure}[H]
    \centering
    \includegraphics[width=0.8\textwidth]{../Resources/GUI_BASIC}
    \caption{Interfaz gr\'afica b\'asica}
\end{figure} 

\section{Selecci\'on de filtro y plantilla}
En esta etapa es posible elegir tanto los par\'ametros de la plantilla que debe cumplir el filtro, como el tipo de aproximaci\'on a utilizar para el dise\~no.
Se muestra a continuaci\'on una captura de la interfaz, donde se detalla cada una de las secciones de inter\'es. Es importante se\~nalar laclara distini\'on estre filtros que deben cumplir con una plantilla de atenuaci\'on y aquellos cuya plantilla esta dada por el retardo de grupo en la banda de paso.
\begin{figure}[H]
    \centering
    \includegraphics[width=0.8\textwidth]{../Resources/ATT_TEMPLATE}
    \caption{Interfaz gr\'afica la selecci\'on de plantilla. Plantilla de atenuaci\'on}
\end{figure} 
\begin{figure}[H]
    \centering
    \includegraphics[width=0.8\textwidth]{../Resources/GD_TEMPLATE}
    \caption{Interfaz gr\'afica la selecci\'on de plantilla. Plantilla de atenuaci\'on}
\end{figure} 

\section{Divisi\'on del sistema por etapas de primer y segundo orden}
\section{Implementaci\'on del circuito}
\section{Dise\~no de un filtro paso a paso}
\end{document}